\section{Conclusion}
\label{sec:conclusion}

We were successful in implementing an AI agent which was able to score an average reward of more than 200 for 100 consecutive episodes and compared different variants of DQN. DQN was applied effectively on this specific problem, and produced successful results.  \\

The DQN variants we tried gave very good results once the hyperparameters were tuned correctly. When implemented the algorithm code in a modularized format, we could easily change the parameters and play with it. Also, minimal tweaks were required for each the DQN variant. Not much of hyperparameter tuning was required across DQN variants. The more advanced Duel Q-learning learner undoubtedly increased the accuracy of the reinforcement learning agent. Given more time and resources, the agent could have been tuned via a more exhaustive grid search. After some literature review, we tried out DQN with Prioritized replay as well. Unfortunately, the results were not promising in comparison to rest of the three approaches. 